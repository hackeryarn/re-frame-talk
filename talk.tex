\documentclass{beamer}

\usepackage{fontspec}
\usepackage{xcolor}

\usefonttheme{serif}
\setmainfont{Charter}


\definecolor{primary}{RGB}{87,130,217}
\definecolor{secondary}{RGB}{99,179,46}

\setbeamercolor{title}{fg=primary}
\setbeamercolor{frametitle}{fg=primary}
\setbeamercolor{itemize item}{fg=primary}

\setbeamercolor{block title example}{fg=secondary}
\setbeamercolor{author}{fg=secondary}
\setbeamercolor{date}{fg=secondary}

\newcommand{\code}[1]{{\color{secondary} \texttt{#1}}}

\title{Clojure SPAs with re-frame}
\author{Artem Chernyak}
\date{2020}

\begin{document}

  \frame{\titlepage}

  \begin{frame}
    \frametitle{Groundwork}
    \begin{itemize}
      \item ClojureScript
      \item Hiccup
      \item React
      \item Reagent
    \end{itemize}
  \end{frame}

  \begin{frame}
    \frametitle{ClojureScript}
    \begin{itemize}
      \item Mostly Clojure
      \item Runs on node
      \item Runs in browser
      \item Reagent
    \end{itemize}
  \end{frame}

  \begin{frame}[fragile]
    \frametitle{Hiccup for HTML}
    \begin{verbatim}
[:html
  [:h1 "Hello World"]]
    \end{verbatim}
    \begin{itemize}
      \item Supports basic html tags
      \item Auto closing tags
      \item Lisp syntax
    \end{itemize}
  \end{frame}

  \begin{frame}[fragile]
    \frametitle{Properties in Hiccup}
    \begin{verbatim}
[:html
  [:h1
    {:class "title"}
    "Hello World"]]

[:html
  [:h1.title "Hello World"]]
    \end{verbatim}
    \begin{itemize}
      \item Supports basic html tags
      \item Auto closing tags
      \item Lisp syntax
    \end{itemize}
  \end{frame}

  \begin{frame}[fragile]
    \frametitle{Clojure in Hiccup}
    \begin{verbatim}
[:ul
  (for [i (range 1 4)]
    [:li i])]
    \end{verbatim}
    \begin{itemize}
      \item Supports basic html tags
      \item Auto closing tags
      \item Lisp syntax
    \end{itemize}
  \end{frame}

  \begin{frame}
    \frametitle{React}
    \begin{itemize}
      \item JavaScript framework
      \item Component based
      \item Highly reusable abstractions
      \item Great performance
    \end{itemize}
  \end{frame}

  \begin{frame}[fragile]
    \frametitle{Clojure in Hiccup}
    \begin{verbatim}
(defn hello [name]
  [:h1  "Hello " name])

[:html
  [hello "World" ]]
    \end{verbatim}
    \begin{itemize}
      \item Use Hiccup
      \item Allows modularizing with components
      \item Work just like function
    \end{itemize}
  \end{frame}

  \begin{frame}[fragile]
    \frametitle{State management}
    \begin{verbatim}
(defn click-count (r/atom 0))

(defn counting-component []
  [:div
   "The atom " [:code "click-count"] " has value: "
   @click-count ". "
   [:input {:type "button" :value "Click me!"
            :on-click #(swap! click-count inc)}]])
    \end{verbatim}
    \begin{itemize}
      \item Uses clojure atom syntax
      \item Allows for dynamic components
      \item Automatically triggers re-render
    \end{itemize}
  \end{frame}

  \begin{frame}[fragile]
    \frametitle{Local state}
    \begin{verbatim}
(defn timer-component []
  (let [seconds-elapsed (r/atom 0)]
    (fn []
      (js/setTimeout #(swap! seconds-elapsed inc)
       1000)
      [:div
       "Seconds Elapsed: " @seconds-elapsed])))
    \end{verbatim}
    \begin{itemize}
      \item Only initializes when neede
      \item Isolated to the component
      \item Can share as needed
      \item Tight re-draw loop
    \end{itemize}
  \end{frame}

  \begin{frame}[fragile]
    \frametitle{Sharing state}
    \begin{verbatim}
(defn atom-input [value]
  [:input {:type "text"
           :value @value
           :on-change #(reset! value
                        (-> % .-target
                            .-value))}])

(defn shared-state []
  (let [val (r/atom "foo")]
    (fn []
      [:div
       [:p "The value is now: " @val]
       [:p "Change it here: " [atom-input val]]])))
    \end{verbatim}
  \end{frame}

  \begin{frame}[fragile]
    \frametitle{Leaky React}
    \begin{verbatim}
(defn list [items]
  [:ul
    (for [item items]
      ^{:key item}
      [:li "Item " name])])
    \end{verbatim}
    \begin{itemize}
      \item Use Hiccup
      \item Allows modularizing with components
      \item Work just like function
    \end{itemize}
  \end{frame}

  \begin{frame}[fragile]
    \frametitle{Passing arguments gotcha}
    \begin{verbatim}
(defn timer-component [event]
  (let [seconds-elapsed (r/atom 0)]
    (fn [event]
      (js/setTimeout #(swap! seconds-elapsed inc)
                     1000)
      [:div
       "Seconds Elapsed since "
       event ": "
       @seconds-elapsed])))
    \end{verbatim}
    \begin{itemize}
      \item Required in both functions
      \item Breaks initial render if missing at top level
      \item Breaks subsequent renders if missing in return
    \end{itemize}
  \end{frame}

  \begin{frame}[fragile]
    \frametitle{Lazy isn't always good}
    \begin{verbatim}
(defn list []
  [:ul
    (doall
      (for [i (range 1 4)]
        [:li i]))])
    \end{verbatim}
    \begin{itemize}
      \item Forces update
      \item Can cause intermittent bugs
    \end{itemize}
  \end{frame}

  \begin{frame}
    \frametitle{Why re-frame?}
    \begin{itemize}
      \item Data oriented
      \item State sharing
      \item Less decisions
    \end{itemize}
  \end{frame}

  \begin{frame}
    \frametitle{The six domains}
    \begin{itemize}
      \item Event dispatch
      \item Event handling
      \item Effect handling
      \item Query
      \item View
      \item DOM
    \end{itemize}
  \end{frame}

  \begin{frame}
    \frametitle{Mental Model}
    \begin{itemize}
      \item Events trigger \code{dispatchers}
      \item Dispatch handlers produce effect maps
      \item Effects modify db and external resources
      \item \code{subscribers} listen to changes on the db
    \end{itemize}
  \end{frame}

  \begin{frame}
    \frametitle{Events}
    \begin{itemize}
      \item User interactions
      \item Other events
      \item Time
    \end{itemize}
  \end{frame}

  \begin{frame}[fragile]
    \frametitle{Dispatching}
    \begin{verbatim}
(defn delete-button [item-id]
 [:button
  {:on-click #(re-frame.core/dispatch
               [:delete-item item-id])}])
    \end{verbatim}
    \begin{itemize}
      \item Just a function
      \item Can take extra parameters
      \item Every dispatch type must have a handler
    \end{itemize}
  \end{frame}

  \begin{frame}[fragile]
    \frametitle{Handling a Dispatch}
    \begin{verbatim}
(defn delete-item
  [{:keys [db]} [_ item_id]]
  {:db (dissoc-in db [:items item-id])})

(rf/reg-event-fx
 :delete-item
 delete-item)
    \end{verbatim}
    \begin{itemize}
      \item Always return effect maps
      \item Effects get automatically triggered
      \item Only effects can modify state
    \end{itemize}
  \end{frame}

  \begin{frame}[fragile]
    \frametitle{Effect handlers}
    \begin{verbatim}
(rf/reg-fx
 :db
 (fn [val]
  (reset! app-db val)))
    \end{verbatim}
    \begin{itemize}
      \item Performs some side effect
      \item Libraries can provide effects
    \end{itemize}
  \end{frame}

  \begin{frame}
    \frametitle{Why?}
    \begin{itemize}
      \item Easy testing
      \item Single point of control
      \item Force separation of concerns
      \item Your own DSL
    \end{itemize}
  \end{frame}

  \begin{frame}[fragile]
    \frametitle{Creating subscribers}
    \begin{verbatim}
(defn query-fn
  [db v]
  (:items db))

(rf/reg-sub
  :query-items
  query-fn)
    \end{verbatim}
    \begin{itemize}
      \item Works like a view over the db
      \item Can combine or modify data
      \item Allows the subscriber to get called from anywhere
    \end{itemize}
  \end{frame}

  \begin{frame}[fragile]
    \frametitle{Using subscribers}
    \begin{verbatim}
(defn items-view
  []
  (let [items @(rf/subscribe [:query-items])]
    [:div (map item-render items)]))
    \end{verbatim}
    \begin{itemize}
      \item Component automatically updates with subscriber
      \item Subscribers can take arguments
    \end{itemize}
  \end{frame}

  \begin{frame}[fragile]
    \frametitle{Subscribers that subscribe}
    \begin{verbatim}
(defn get-item-fn
  [items [_ id]]
  (get items id))

(rf/reg-sub
  :get-item
  :<- [:items]
  query-fn)
    \end{verbatim}
    \begin{itemize}
      \item Automatically update when underlying subscribers change
      \item Allows incrementally computing computations
    \end{itemize}
  \end{frame}

  \begin{frame}[fragile]
    \frametitle{Coeffects}
    \begin{verbatim}
(rf/reg-cofx
  :now
  (fn [coeffects _]
     (assoc coeffects :now (js.Date.))))

(rf/reg-event-fx
  :dont-care
  [(inject-cofx :now)]
  (fn [{:keys [db now]}]
    ...))
    \end{verbatim}
    \begin{itemize}
      \item Allows getting things from the outside re-frame
      \item Keeps re-frame functions pure
      \item easy testing
    \end{itemize}
  \end{frame}

  \begin{frame}[fragile]
    \frametitle{Make testing easier}
    \begin{verbatim}
(defmacro reg-sub [id & args]
  `(do
     (def ~(-> id name symbol) ~(last args))
     (rf/reg-sub
      ~id
      ~@args)))
    \end{verbatim}
    \begin{itemize}
      \item Automatically creates a function for testing
      \item Keeps function and registration together
    \end{itemize}
  \end{frame}

  \begin{frame}
    \frametitle{Always namespace}
    \begin{itemize}
      \item Create with \code{::event}
      \item Reference through named import \code{:my/event}
      \item Clean organization, easy navigation
    \end{itemize}
  \end{frame}

  \begin{frame}
    \frametitle{Database recommendations}
    \begin{itemize}
      \item Keep things in maps by unique id
      \item Transform as needed in subscribers
      \item Do not copy whole objects, use the ids
    \end{itemize}
  \end{frame}

\end{document}
