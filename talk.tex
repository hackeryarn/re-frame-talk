\documentclass{beamer}

\usepackage{fontspec}
\usepackage{xcolor}

\usefonttheme{serif}
\setmainfont{Charter}


\definecolor{primary}{RGB}{87,130,217}
\definecolor{secondary}{RGB}{99,179,46}

\setbeamercolor{title}{fg=primary}
\setbeamercolor{frametitle}{fg=primary}
\setbeamercolor{itemize item}{fg=primary}

\setbeamercolor{block title example}{fg=secondary}
\setbeamercolor{author}{fg=secondary}
\setbeamercolor{date}{fg=secondary}

\newcommand{\code}[1]{{\color{secondary} \texttt{#1}}}

\title{Clojure SPAs with re-frame}
\author{Artem Chernyak}
\date{2020}

\begin{document}

  \frame{\titlepage}

  \begin{frame}
    \frametitle{Groundwork}
    \begin{itemize}
      \item ClojureScript
      \item Hiccup
      \item React
      \item Reagent
    \end{itemize}
  \end{frame}

  \begin{frame}
    \frametitle{ClojureScript}
    \begin{itemize}
      \item Mostly Clojure
      \item Runs on node
      \item Runs in browser
      \item Reagent
    \end{itemize}
  \end{frame}

  \begin{frame}[fragile]
    \frametitle{Hiccup for HTML}
    \begin{verbatim}
[:html
  [:h1 "Hello World"]
    \end{verbatim}
    \begin{itemize}
      \item Supports basic html tags
      \item Auto closing tags
      \item Lisp syntax
    \end{itemize}
  \end{frame}

  \begin{frame}[fragile]
    \frametitle{Clojure in Hiccup}
    \begin{verbatim}
[:ul
  (for [i (range 1 4)]
    [:li i])]
    \end{verbatim}
    \begin{itemize}
      \item Supports basic html tags
      \item Auto closing tags
      \item Lisp syntax
    \end{itemize}
  \end{frame}

  \begin{frame}
    \frametitle{React}
    \begin{itemize}
      \item JavaScript framework
      \item Component based
      \item Highly reusable abstractions
      \item Great performance
    \end{itemize}
  \end{frame}

\end{document}
